\documentclass{article}
\usepackage[utf8]{inputenc}
\usepackage[spanish]{babel}
\usepackage{listings}
\usepackage{graphicx}
\graphicspath{ {images/} }
\usepackage{cite}

\begin{document}

\begin{titlepage}
    \begin{center}
        \vspace*{1cm}
            
        \Huge
        \textbf {Actividad 1 Proyecto Final}
            
        \vspace{0.5cm}
        \LARGE
        Informatica II
            
        \vspace{1.5cm}
            
        \textbf{Daniela Andrea Gallego Díaz\\ Manuela Gutiérrez Rodríguez }
            
        \vfill
            
        \vspace{0.8cm}
            
        \Large
        Despartamento de Ingeniería Electrónica y Telecomunicaciones\\
        Universidad de Antioquia\\
        Medellín\\
        Marzo de 2021
            
    \end{center}
\end{titlepage}

\tableofcontents
\newpage
\section{Introducción}\label{intro}
En este documento se plasmarán cada una de las ideas para la realización de nuestro proyecto final. Cada opción que consideramos para el videojuego está descrita detalladamente para dar a entender lo que posiblemente crearemos en este proyecto. Cabe aclarar que tenemos multiples opciones ya que quisimos documentar cada una de las ideas que ibamos teniendo para luego elegir la que nos llamara más la atención. Por otra parte, consideramos tener diversas opciones para elegir conscientemente la que más se adapte a nuestros gustos y conocimientos, para crear un proyecto innovador y de calidad.


\section{Idealización del videojuego} \label{contenido}



\subsection{Primera opción: demon and angel}
Desarrollar un video juego para dos ususarios que consiste en dos personajes, uno similar a un ángel y el otro a un demonio, antes de iniciar el juego, el usuario va a tener la posibilidad de personalizar ambos personajes ( cambiar sus vestuarios, elegir las expresiones de sus rostros e incluso el color de los personajes). 
\vspace{0.2 cm}

El juego consiste en que ambos personajes deben enfrentarse a varios niveles, obtener gemas y hacerlo en el menor tiempo posible, además de evitar morir y enfrentarse a ciertos enemigos que tratarán de impedir que logren su objetivo. 
\vspace{0.2 cm}

La interfaz va a estar compuesta de tal forma que en primer lugar aparezca un menú para iniciar el juego, luego se muestra la pestaña en la cual se va a poder personalizar los personajes y por último va a aparecer en el nivel que está. Dentro del juego, va a haber una pestaña que le permita al usuario volver al inicio si lo desea. Este juego es para dos personas, el personaje del ángel se maneja con las teclas w,a,s,d y el personaje del demonio con las teclas de las flechas.

\subsection{Segunda opción: the again game}
Este videojuego sigue un concepto que consiste en aparentemente repetir el mismo nivel una y otra vez, sin embargo, cada nivel tiene un acertijo y una forma diferente de solucionarse. El nivel se gana cuando el participante logra completar el desafío con las pistas que se le dan, aún así tiene el riesgo de morir. 

En este juego participan dos personajes, el personaje principal empieza a recorrer su camino completamente solo y su objetivo es llegar donde el segundo personaje y rescatarlo de la muerte.

La interfaz de este juego es bastante sencilla, ya que su diseño va a ser muy similar para cada nivel, el usuario puede acceder que nivel jugar siempre y cuando haya logrado superarlos todos (se piensa en aproximadamente de 15 a 20 niveles). Cabe aclarar que este juego es unicamente para una sola persona, el personaje principal se maneja con las teclas de flechas que trae por defecto cada teclado y el personaje secundario con las teclas w,a,s,d.

\subsection{Tercera opción: running in the mall}
El juego consiste en un personaje principal que es mujer y debe correr a través de un centro comercial, esta mujer debe escapar del guardia de seguridad que la persigue, mientras lo intenta lograr puede subir por las escaleras electricas, saltar obstáculos y evitarlos; además, puede ir recogiendo monedas que después serán útiles a la hora de querer personalizar el personaje. El juego termina cuando la mujer es atrapada por el guardia.
\vspace{0.2 cm}

Existen elementos que dan poderes especiales, por ejemplo si se encuentra una tarjeta de crédito se duplican sus monedas, si encuentra unos tacones puede saltar más alto de lo normal y por último si se encuentra unas alas puede volar por el techo por unos 10 segundos y recoger monedas sin la necesidad de enfrentarse a obstáculos. 
\vspace{0.2 cm}

La interfaz de este videojuego consiste en un menú principal que muestra cuantos puntos ha logrado el personaje, una pestaña para comprar vestuario al personaje (con las monedas que se van ganando) y la posibilidad de inciciar y parar juego cuando se desee. Este personaje se va a manejar con las teclas de flechas.

\subsection{Cuarta opción: the master key}
El juego consiste en un personaje principal el cual tendrá que pasar por una serie de laberintos (se hacen más difíciles a medida que sube de nivel). Para pasar de un nivel a otro deberá encontrar una llave, la cual está en posesión de uno de los enemigos que deberá derrotar, además de encontrar la puerta que lleva al siguiente nivel.
\vspace{0.2 cm}


\subsection{Quinta opción: El teleférico de la muerte}
El juego consiste en una especie de teleférico que al inicio recoge a 4 personas, el teleférico puede avanzar hacia adelante y hacia atrás, este se enfrenta a obstáculos y si no logra evadirlos una persona del teleférico cae, al final entre más personas logre salvar más puntos recibe, sino logra salvar a nadie, pierde el juego y no puede avanzar al siguiente nivel.

\subsection{Sexta opción: carrera de autos}
En este juego habrá varios personajes para elegir, cada uno con una especialidad, también varias opciones de automóviles para elegir al igual que paracaidas, los cuales se irán desbloqueando a medida que subes de nivel o puedes comprarlos con las monedas que ganes en cada carrera. El objetivo del juego es simple, ser el primero en llegar a la meta pasando con éxito todos los obstáculos en el camino.


\end{document}
